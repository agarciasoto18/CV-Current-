\documentclass{resume} % Use the custom resume.cls style

\usepackage[left=0.95in,top=0.94in,right=1in,bottom=0.90in]{geometry} % Document margins
\usepackage{pifont} 
\usepackage[colorlinks,urlcolor=blue, linkcolor=blue]{hyperref}
\usepackage{graphicx}                        % Required to draw the flag
\usepackage[document]{ragged2e}

\begin{document}

\vspace{-1.0in}
{\LARGE\bf Aylin Garc\'\i a Soto} \\% Your name at the top 
\vphantom{Danny Phantom}
{\href{https://aylingarciasoto.com}{Website} \ding{169} \href {https://orcid.org/0000-0001-9828-3229}{ORCID}}   \ding{169}  {\href{http://www.linkedin.com/in/aylin-garcia-soto-02b24a3a}{LinkedIn}} \ding{169} {\href {mailto:aylin.garcia.soto.gr@dartmouth.edu}{E-mail} \ding{169}  West Lebanon, NH} \hfill
% \begin{figure}[h]
% \vspace{.3in}
% \hfill\smash{\includegraphics[width=2.5cm,keepaspectratio]{IMG2514.jpeg}}
% \end{figure}

% \vspace{-.5in}
\begin{rSection}{Education}

{\bf Dartmouth College} \hfill Hanover, NH \hfill {September 2019 - Present} \\ 
{\textbf{\textit{Doctor of Philosophy}}} in Astrophysics

{\bf Wesleyan University} \hfill Middletown, CT \hfill {September 2014 - May 2018} \\ 
{\textbf{\textit{Bachelor of Arts (with Honors)}}} in Astronomy and Physics  \hfill {GPA: 3.55/4.00}
\end{rSection}
\begin{rSection}{Research Experience}

{\bf Research Assistant}
{\hfill Dartmouth College} \\
{\sl with Professor Elisabeth Newton} \hfill September 2019 - Present
\begin{itemize} \itemsep -6pt
\item Analysis of optical photometry data from the Transiting Exoplanet Survey Satellite (TESS) of active M dwarfs and operation and reduction of optical spectra on the MDM Hiltner 2.4m spectrograph, OSMOS and on the MDM McGraw-Hill 1.3m spectrograph, MODSPEC.
\item Publishing a first-author paper in 2023.
\end{itemize}

{\bf TESS Research Support Associate}
{\hfill Massachusetts Institute of Technology} \\
{\sl with Professor Sara Seager} \hfill September 2018 - August 2019
\begin{itemize} \itemsep -6pt
\item Vetting TESS data and creation of corresponding documentation to contribute to planet candidate identification and TESS objects of interests. 
\item Management of several public-facing MIT TESS websites and creation of the first TESS science conference website.
\item Contributed to \texttt{TESS-ExoClass (TEC)} - a software that vets exoplanets from the TESS NASA/MIT mission.
\end{itemize}

{\bf Research Assistant}
{\hfill Wesleyan University} \\
{\sl with Professor William Herbst} \hfill June 2018 - April 2020
\begin{itemize} \itemsep -6pt
\item Reduction and analysis of \textit{VRIJHK} photometry of the V 582 Mon/KH 15D system.
\item Comparison of the new model for the system to the current photometric data, in order to constrain the properties of circumbinary disk.
\item Publishing a first-author paper in 2020
\end{itemize}

{\bf Honors Thesis Research} \hfill Wesleyan University  \\
{\sl with Professor Edward Moran} \hfill May 2016 - May 2018 
\begin{itemize} \itemsep -6pt % Reduce space between items
\item Calibration of data from ROSAT using NASA's High Energy Astrophysics Science Archive Research Center's Ftools and operation of the MDM Hiltner 2.4m telescope.
\item Implementation of numerical methods to perform aperture photometry and a compilation of light-curves, using python, to search for AGNs that have varied dramatically in the long-term.
\end{itemize}

{\bf Banneker and Aztl\'an Institutes}
{\hfill Harvard-Smithsonian Center for Astrophysics} \\
{\sl with Dr. Joseph E. Rodriguez} \hfill June 2017 - August 2017
\begin{itemize} \itemsep -6pt
\item Collection, reduction, and fitting of photometric observations of eclipsing binaries with the FLWO 1.2m telescope and the Harvard University 0.4m Clay telescope, MaxIm DL, and AstroImageJ, respectively.
\item Participation in graduate-level courses in various Astronomical topics.
\end{itemize}

{\bf Keck Northeastern Astronomy Consortium} \hfill Williams College  \\
{\sl with Dr. Steven Souza} \hfill June 2015 - October 2015 
\begin{itemize} \itemsep -6pt % Reduce space between items
\item Masking of data to account for the deteriorating CCD camera filter, and processing the accumulated data using AstroImageJ and the Aperture Photometry Tool 2.1.8.
\item Operation of the Hopkins 24-inch Telescope to image young open star clusters.
\end{itemize}


\end{rSection}

\begin{rSection}{PUBLICATIONS FIRST AUTHOR}
Garc{\'\i}a Soto, A., Newton, E.~R., Douglas, S.~T., Burrows, A., Kesseli, A.~Y.\ 2023. \ \textit{The Astronomical Journal} \textbf{165, 192} \href{https://ui.adsabs.harvard.edu/abs/2023AJ....165..192G/abstract}{doi:10.3847/1538-3881/acc2ba}

Garc{\'\i}a Soto, A., Ali, A., Newmark, A., Herbst, W., Windemuth, D., Winn, J.~N.\ 2020.\ \textit{The Astronomical Journal}, \textbf{159, 135.} \href{https://ui.adsabs.harvard.edu/abs/2020AJ....159..135G/abstract}{doi:10.3847/1538-3881/ab6efd}
\end{rSection}

\begin{rSection}{PUBLICATIONS CO-AUTHOR}
Guerrero, N. M., Seager, S., Huang, C. X., Vanderburg, A., \textbf{Garcia Soto, A.}, and  100 collegues \ 2021, \textit{The Astrophysical Journal Supplement Series}, \textbf{254, 39.} 

Rowden, P., Borkovits, T., et al. 2020.\  \textit{The Astronomical Journal}, \textbf{160, 76.}

Parviainen, H., Palle, E., Zapatero-Osorio, M.~R., et al.\ 2020, \textit{Astronomy \& Astrophysics}, \textbf{ 633, A28.} 

Huber, D., and 141 colleagues \ 2019.\  \textit{The Astronomical Journal}, \textbf{157, 245.}

Rodriguez, J.~E., and 73 colleagues \ 2019.\ \textit{The Astronomical Journal}, \textbf{157, 191.}

Dragomir, D., and 50 colleagues \ 2019.\ \textit{The Astrophysical Journal}, \textbf{875, L7.}

\end{rSection}

\begin{rSection}{SELECTED PROFESSIONAL PRESENTATIONS}
\sloppy ``Contemporaneous Observations of H{\ensuremath{\alpha}} Luminosities and Photometric Amplitudes for M dwarfs." \textbf{Garc{\'\i}a Soto, A.}, Newton, E.~R., Burrows, A.~D., Sweeney, T., Douglas, S.~T., Kesseli, A.~Y.\ 2022.\ Cambridge Workshop on Cool Stars, Stellar Systems, and the Sun. \href{https://zenodo.org/record/7523687#.ZEqKAuzMI-Q}{doi:10.5281/zenodo.7523687}

``Different Observational Properties of the Leading and Trailing Edges of the KH 15D Circumbinary Ring", \textbf{Garc{\'\i}a Soto, A}, Ali, A, Newmark, A, Herbst, W, Windemuth, D, and Winn, J., \textbf{233rd Meeting of the American Astronomical Society}, Seattle, WA, January 2019. \href{https://ui.adsabs.harvard.edu/abs/2019AAS...23315412G/abstract}{2019AAS...23315412G}

``Photometric Follow-up of Eclipsing Binary Candidates from KELT and Kepler", \textbf{Garc{\'\i}a Soto, A}, Rodriguez, J. E., and Bieryla, A., \textbf{231st Meeting of the American Astronomical Society}, National Harbor, MD, January 2018. \href{https://ui.adsabs.harvard.edu/abs/2018AAS...23124412G/abstract}{2018AAS...23124412G}

``Uncovering extreme AGN variability in serendipitous X-ray source surveys", E. C. Moran, \textbf{Garcia Soto, A.}, LaMassa, S., Urry, M., \textbf{231st Meeting of the American Astronomical Society}, National Harbor, MD, January 2018. \href{https://ui.adsabs.harvard.edu/abs/2018AAS...23123814M/abstract}{2018AAS...23123814M}

``Hour-Scale Variability in NGC 663 and NGC 1960", Souza, S. P., \textbf{Garcia Soto, A.}, and Wong H. , \textbf{228th Meeting of the American Astronomical Society}, San Diego, CA, June 2016. \href{https://ui.adsabs.harvard.edu/abs/2016AAS...22831903S/abstract}{2016AAS...22831903S}\\
\end{rSection}

\begin{rSection}{HONORS/AWARDS}
\textbf{The Selamawit Tsehaye Teaching Award:} \hfill June 2021\\
\textit{awarded to teaching assistants who show dedication to teaching.}\\

\textbf{Honorable Mention for the NSF GRFP:} \hfill March 2021\\
\textit{awarded by the NSF to students with strong applications for the graduate fellowship.}\\

\textbf{Littell Prize: } \hfill May 2018\\
\textit{awarded by the Wesleyan astronomy department for excellence in advanced astronomy classes.}\\

\textbf{McNair Scholar:} \hfill January 2015-May 2018\\
\textit{awarded to undergraduate students with the aspiration of pursuing a PhD.}\\
\textbf{NASA CT Space Grant Scholarship:}	\hfill September 2015\\
\textit{awarded to undergraduates in science for assistance with financial aid.}\\
\textbf{Janey Scholar:}	\hfill September 2014-June 2022\\
\textit{awarded high-achieving, low-income students pursuing higher education.}\\
% \textbf{Sun Life Rising Scholar Scholarship:}	\hfill Sept. 2014\\
% \textit{awarded to high schools students with leadership, communal, and studious qualities.
% }
\end{rSection}

\begin{rSection}{Educational Outreach Experience}
{\bf Dartmouth Public Observing} \hfill{Dartmouth College}\\
{\sl Organizer/Volunteer/Telescope Operator} \hfill September 2019 - Present
\begin{itemize} \itemsep -5pt
\item Operation of the 14-inch Meade telescope for several nights while teaching the public about space and my research.
\item NEW PO TBD.
\end{itemize}

{\bf The Montshire Museum of Science} \hfill{Norwich, Vermont}\\
{\sl Volunteer} \hfill September 2019 - Present
\begin{itemize} \itemsep -5pt
\item Co-leading and creating activities for children in the Vermont and New Hampshire area on the yearly Astronomy Science Day.
\end{itemize}

{\bf Dartmouth Many Mentors} \hfill{Dartmouth College}\\
{\sl Volunteer} \hfill September 2019 - 2021
\begin{itemize} \itemsep -5pt
\item Volunteering to visit and help with science projects at high schools picked by Many Mentors.
\end{itemize}

{\bf Astronomical Pedagogy (Kids' Night and Space Night)} \hfill{Wesleyan University}\\
{\sl Presenter/Volunteer} \hfill September 2015 - May 2018
\begin{itemize} \itemsep -5pt
\item Organization of Astronomical presentations--16 Kids' Nights and 4 Space Nights--operation of the 16-inch Meade Telescope and the 24-inch Perkin Telescope.
\item Creation of hands-on activities for children and/or for the Middletown community and their families to help improve Scientific literacy.
\end{itemize}

{\bf Wesleyan Science Outreach} \hfill{Wesleyan University} \\
{\sl Co-Coordinator/Teaching Assistant/Volunteer} \hfill September 2014 - January 2018
\begin{itemize} \itemsep -5pt
\item Co-instruction of the pedagogical course CHEM242, including the organization of the class schedule and preparation of lesson plans and demos.
\item Preparation of the students or club leaders responsible for creating lesson plans, and the recruitment of volunteers for school visits and Science Saturday.
\end{itemize}
\end{rSection}

\begin{rSection}{Other Experience}
{\bf Graduate Student Council} \hfill{Dartmouth College} \\
{\sl Representative of the Physics and Astronomy Department}
\hfill October 2020 - July 2022
\begin{itemize} \itemsep -5pt
\item Co-organization of events for the student life committee, including re-instituting Mental Health Monday.
\item Co-organization of events for the Committee of Addressing Racism and Equity. This is a very new committee and the first meeting has not happened as of this application deadline.
\end{itemize}
\end{rSection}
% \begin{rSection}{MEMBERSHIPS}
% \textbf{Junior Membership:} American Astronomical Society\\
% \textbf{Student Membership:} American Physical Society \\
% \end{rSection}
\begin{rSection}{SOFTWARE}
\begin{itemize}
\item\url{https://github.com/agarciasoto18/starrotate}
\item \url{https://github.com/christopherburke/TESS-ExoClass}
\end{itemize}

\end{rSection}
\begin{rSection}{SKILLS/LANGUAGES}
\textbf{Advanced}: Python programming, Collaboration, Numerical Methods, \LaTeX \& Photoshop \\
\textbf{Proficient}:  HTML, revealJS, UNIX, GitHub, IRAF, PypeIt \\ 
\textbf{Fluent}: English, Spanish \\
% \textbf{Proficient}: French
\textbf{Beginner}: French, Italian, Portuguese
\end{rSection}

\end{document}